%%%%%%%%%%%%%%%%%%%%%%%%%%%%%%%%%%%%%%%%%
% Plasmati Graduate CV
% LaTeX Template
% Version 1.0 (24/3/13)
%
% This template has been downloaded from:
% http://www.LaTeXTemplates.com
%
% Original author:
% Alessandro Plasmati (alessandro.plasmati@gmail.com)
%
% License:
% CC BY-NC-SA 3.0 (http://creativecommons.org/licenses/by-nc-sa/3.0/)
%
% Important note:
% This template needs to be compiled with XeLaTeX.
% The main document font is called Fontin and can be downloaded for free
% from here: http://www.exljbris.com/fontin.html
%
%%%%%%%%%%%%%%%%%%%%%%%%%%%%%%%%%%%%%%%%%

%----------------------------------------------------------------------------------------
%	PACKAGES AND OTHER DOCUMENT CONFIGURATIONS
%----------------------------------------------------------------------------------------

\documentclass[a4paper,10pt]{article} % Default font size and paper size

\usepackage{fontspec} % For loading fonts
\defaultfontfeatures{Mapping=tex-text}
\setmainfont[SmallCapsFont = Fontin SmallCaps]{Fontin} % Main document font

\usepackage{xunicode,xltxtra,url,parskip} % Formatting packages

\usepackage[usenames,dvipsnames]{xcolor} % Required for specifying custom colors

\usepackage[big]{layaureo} % Margin formatting of the A4 page, an alternative to layaureo can be \usepackage{fullpage}
% To reduce the height of the top margin uncomment: \addtolength{\voffset}{-1.3cm}

\usepackage{hyperref} % Required for adding links	and customizing them
\definecolor{linkcolour}{rgb}{0,0.2,0.6} % Link color
\hypersetup{colorlinks,breaklinks,urlcolor=linkcolour,linkcolor=linkcolour} % Set link colors throughout the document

\usepackage{titlesec} % Used to customize the \section command
\titleformat{\section}{\Large\scshape\raggedright}{}{0em}{}[\titlerule] % Text formatting of sections
\usepackage{hyperref}
\titlespacing{\section}{0pt}{3pt}{3pt} % Spacing around sections

\begin{document}

\pagestyle{empty} % Removes page numbering

\font\fb=''[cmr10]'' % Change the font of the \LaTeX command under the skills section

%----------------------------------------------------------------------------------------
%	NAME AND CONTACT INFORMATION
%----------------------------------------------------------------------------------------
\begin{samepage}
\par{\centering{\Huge Marllon Cristian \textsc{Alves}}\bigskip\par} % Your name

\section{Dados Pessoais}

\begin{tabular}{rl}
\textsc{Lugar e Data de Nascimento:} & Vitorino - Paraná | 23 de Outubro de 1991 \\
\textsc{Endereço:} & Rua Aster 354, Holambra, São Paulo, Brasil\\
\textsc{Telefone:} & +55 19 9 9389-7959\\
\textsc{E-mail:} & \href{mailto:camarllon@gmail.com}{camarllon@gmail.com}
\end{tabular}

%----------------------------------------------------------------------------------------
%	WORK EXPERIENCE 
%----------------------------------------------------------------------------------------

\section{Experiência Profissional}

\begin{tabular}{r|p{11cm}}

\emph{Atualmente} & Analista de Sistemas em \textsc{Centro de Pesquisa e Desenvolvimento em Telecomunicações (CPqD)}, Campinas - São Paulo \emph{} \\
\textsc{FEV 2014} & \emph{Análise, Projeto e Implementação}\\ 
                  & \footnotesize{Análise e desenvolvimento de sistemas de controle de planta utilizando linguagem \textit{Java} e tecnologias como \textit{CORBA} e \textit{SNMP}. Atuação em implementação de interfaces com o usuário utilizando framework \textit{Primefaces}, criação de processos implementados sobre tecnologia \textit{EJB} e \textit{Web Services} para realização de procedimentos em lote e manutenção do sistema. Participação efetiva em projeto e implementação de sistema distribuído para verificação de falhas e otimização de malha no setor elétrico utilizando tecnologias e componentes Microsoft (C\#.net, WCF(Windows Communication Foundation), Windows Services), bem como integração com Sistema de Supervisão e Aquisição de Dados (SCADA - DNP3). }\\
\multicolumn{2}{c}{} \\

%------------------------------------------------

\emph{AGO 2013 - DEZ 2015} & Docente em \textsc{Faculdade de Jaguariúna}, Jaguariúna - São Paulo \emph{} \\
& \emph{Docência}\\ 
				& \footnotesize{Docência em ensino superior no curso de \textit{Ciência da Computação} com as disciplinas de \textit{Sistemas Distribuídos} e \textit{Paradigmas de Linguagens de Programação}. Lecionando também a disciplina de \textit{Microcontroladores} para o curso de \textit{Tecnologia em Automação Industrial}. Ambas as disciplinas envolvem conhecimento em lógica/programação, desenvolvimento de aplicações utilizando \emph{Linguagem C++/Assembly}, compreensão dos paradigmas: estruturado, orientado a objeto, funcional e lógico.}\\
\multicolumn{2}{c}{} \\

%------------------------------------------------

\emph{AGO 2013 - FEV 2014} & Analista Desenvolvedor de Sistemas em \textsc{Faculdade de Jaguariúna}, Jaguariúna - São Paulo \emph{}\\
& \emph{Desenvolvimento e Manutenção}\\ 
                  & \footnotesize{Análise e desenvolvimento de sistemas institucionais utilizando linguagem \textit{ PHP }, \textit{ C\# } e \textit{Java}. Desenho de interfaces web utilizando frameworks de padrão MVC \textit{Ember.JS}, linguagem \textit{HTML5} e framework \textit{CSS Twitter Bootstrap 3}. Manutenção de sistemas de \textit{CMS}, \textit{E-mail Marketing} e \textit{SMS Messaging}.}\\
\multicolumn{2}{c}{} \\

%------------------------------------------------

\textsc{ABR 2012 - SET 2012} & Iniciação Científica em \textsc{Faculdade de Jaguariúna}, Jaguariúna - São Paulo \emph{}\\
                           & \footnotesize{Pesquisa e desenvolvimento de artigo científico intitulado ''Análise comparativa entre abordagem conexionista e heurística evolucionária como ferramenta de auxílio ao processo de tomada de decisão no mercado financeiro brasileiro'' que procurava entender e comparar do ponto de vista financeiro, duas abordagens(\textit{I.e: Heurística Evolucionária e Solução Conexionista}) utilizadas como ferramentas de tomada de decisão para analistas financeiros.}\\
\multicolumn{2}{c}{} \\

%------------------------------------------------

\textsc{ABR 2011 - SET 2011} & Iniciação Científica em \textsc{Faculdade de Jaguariúna}, Jaguariúna - São Paulo \emph{}\\
                           & \footnotesize{Pesquisa e desenvolvimento de artigo científico intitulado ''Sistema Imunológico Artificial aplicado na solução do problema do caixeiro viajante'', no qual se procurava entender a relação entre a classificação de problemas NP e NP-Completo. Implementação de um Sistema Imunológico Artificial juntamente com uma instância do problema do caixeiro viajante (Vinda da TSPLIB). Após o estudo dos conceitos o algoritmo em questão foi codificado utilizando linguagem \textit{Java}.}\\
\multicolumn{2}{c}{} \\

\end{tabular}
\end{samepage}

\begin{tabular}{r|p{11cm}}
%------------------------------------------------

\textsc{DEZ 2010 - JAN 2011} & Estágio Supervisionado em \textsc{Stefanini IT Solutions}, Jaguariúna - São Paulo \emph{}\\
                           & \footnotesize{Desenvolvimento e manutenção de aplicações comerciais. Tecnologias envolvendo Mainframes (\emph{COBOL e Padrões de Projeto}) e integração com linguagens de alto nível como \textit{Java} e \textit{ C\# }.}\\
\multicolumn{2}{c}{} \\
\end{tabular}

%----------------------------------------------------------------------------------------
%	EDUCATION
%----------------------------------------------------------------------------------------

\section{Educação}

\begin{tabular}{rl}	
\textsc{FEV} 2015 & Pós Graduação em \textsc{Gestão de Segurança da Informação} \\& \normalsize\textbf{Faculdade de Jaguariúna}, Jaguariúna\\
%& Thesis: ``Money is the Root of All Evil - Or is it?'' | \small Advisor: Prof. James \textsc{Smith}\\
%&\normalsize \textsc{Gpa}: 8.0/9.0\hyperlink{grds}{\hfill | \footnotesize Detailed List of Exams}\\
&\\

%------------------------------------------------

\textsc{DEZ} 2012& Graduação em \textsc{}\textsc{Ciência da Computação} \\& \normalsize\textbf{Faculdade de Jaguariúna}, Jaguariúna\\
%& Heavily specialized in mundane paperwork | \small Advisor: Stefano \textsc{Bonini}\\
%&\normalsize \textsc{Gpa}: 7.5/9.0 \hyperlink{grds_usc}{\hfill| \footnotesize Detailed List of Exams}\\
&\\

%------------------------------------------------

%\textsc{Fall} 2008 & Exchange Semester at \textbf{University of Southern California}, Los Angeles\\
%& \textsc{Gpa}: 8.0/9.0 \hyperlink{grds_usc}{\hfill| \footnotesize Detailed List of Exams}\\
%&\\

%------------------------------------------------

%\textsc{July} 2006& \textbf{Liceo Classico ``E. Duni''}, Matera | Final Grade: 100/100
\end{tabular}


%----------------------------------------------------------------------------------------
%	PUBLICAÇÕES
%----------------------------------------------------------------------------------------
\section{Publicações}

ALVES, C. M.; \textbf{Sistema Imunológico Artificial aplicado na solução do problema do caixeiro viajante}. Em: Congresso Nacional de Iniciação Científica - CONIC/SEMESP, 2011, Santos - SP. Anais CONIC/SEMESP, 2010.


%----------------------------------------------------------------------------------------
%	SCHOLARSHIPS AND ADDITIONAL INFO
%----------------------------------------------------------------------------------------

\section{Cursos e Certificações}

\begin{tabular}{rl}
  \textsc{JUL} 2012 - \textsc{OUT} 2012 & CCNA Exploration \footnotesize(240 Hrs)\normalsize\\
  \textsc{OUT} 2007 - \textsc{AGO} 2008 & Advanced Programming Course \footnotesize(160 Hrs)\normalsize\\

%  \textsc{June} 2010 & {\textsc{Gmat}\textregistered}\setmainfont[SmallCapsFont=Fontin SmallCaps]{Fontin-Regular}: 730 (\textsc{q:50;v:39}) 96\textsuperscript{th} percentile; \textsc{awa}: 6.0/6.0 (89\textsuperscript{th} percentile)
\end{tabular}

%----------------------------------------------------------------------------------------
%	LANGUAGES
%----------------------------------------------------------------------------------------

\section{Idiomas}

\begin{tabular}{rl}
\textsc{Inglês:} & Intermediário\\
\end{tabular}

%----------------------------------------------------------------------------------------
%	COMPUTER SKILLS 
%----------------------------------------------------------------------------------------

\section{Competências em Computação}

  Linguagens e Frameworks:  \textsc{Java EE (Android SDK, Hibernate, Spring e Primefaces)}, \textsc{PHP (Zend Framework I e II, Doctrine 2)}, \textsc{C\#.net (Entity Framework, MVC.NET)},\textsc{Angular.JS1 (Ionic)}, \textsc{Delphi - Object Pascal}, \textsc{Assembly (x86 -- At\&T/Intel)}, \textsc{Python}, \textsc{HTML}, \textsc{CSS}, \textsc{Javascript (jQuery, jQuery UI)}, \textsc{Shell Script - BASH, ZSH}, \textsc{C/C++}, \textsc{Perl}, \textsc{Prolog} e \textsc{LISP - Common Lisp}. 
\\
\\
  Sistemas Gerenciadores de Bancos de Dados (SGDBs):  \textsc{Oracle}, \textsc{MYSql}, \textsc{MARIADB}, \textsc{H2} e \textsc{SQLite}. 
\\
\\
  Sistemas Operacionais: \textsc{Arch Linux}, \textsc{Slackware 12/13} e \textsc{Debian}.
%  {\fb}\setmainfont[SmallCapsFont=Fontin SmallCaps]{Fontin-Regular}\\


%----------------------------------------------------------------------------------------
%	INTERESTS AND ACTIVITIES
%----------------------------------------------------------------------------------------

\section{Interesses e Atividades}

Programação, Estatística, Processos, Inteligência Artificial, Geometria Computacional, Teste de Intrusão, Análise de Malware, Pesquisa de Vulnerabilidades, Lógica,  Filosofia, Basquete, Música. %{\fb}\setmainfont[SmallCapsFont=Fontin SmallCaps]{Fontin-Regular}\\
%----------------------------------------------------------------------------------------


%----------------------------------------------------------------------------------------
%	INFORMAÇÕES COMPLEMENTARES
%----------------------------------------------------------------------------------------

\section{Informações Complementares}

\begin{itemize}
  \item Professor Avaliador de Trabalhos de Conclusão de Curso (TCC) no evento \textbf{XV ENIC - Encontro de Iniciação Científica} da Faculdade de Jaguariúna em 2015;
  \item Representante da Faculdade de Jaguariúna na \textbf{ROBOCORE} (Competição Nacional de Robótica) em 2011, obtendo terceira colocação na categoria Sumo Lego 1Kg;
  \item Técnico dos times da Faculdade de Jaguariúna classificados para Final Brasileira na \textbf{Maratona de Programação} (ICPC) 2015;  
  \item Instrutor em oficina sobre Análise Forense (Hands On) no evento \textbf{COMPUFAJ} da Faculdade de Jaguariúna em 2014;
  \item Técnico dos times da Faculdade de Jaguariúna que participaram da \textbf{Maratona de Programação} (ICPC) 2014;  
  \item Representante da Faculdade de Jaguariúna na \textbf{Maratona de Programação} (ICPC) em 2010 e 2011;  
  \item Representante da Faculdade de Jaguariúna na \textbf{ROBOCORE} (Competição Nacional de Robótica) em 2011, obtendo terceira colocação na categoria Sumo Lego 1Kg;
  \item Participante do \textbf{Desafio Sebrae} de Empreendedorismo em 2009 e 2011;
  \item Participante da \textbf{Maratona de Programação} Interna do curso de Ciência da Computação da Faculdade de Jaguariúna em 2011;
  \item Representante do curso de Ciência da Computação no evento \emph{FAJ Aberta} em 2011;
  \item Autor do blog \textbf{Marllon's blog} (\url{https://marllonalves.com.br});
  {\fb}\setmainfont[SmallCapsFont=Fontin SmallCaps]{Fontin-Regular}\\
\end{itemize}

%----------------------------------------------------------------------------------------



\end{document}
